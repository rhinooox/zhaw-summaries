\section{Zahlensysteme}
\textbf{Umrechnen von Dezimalzahlen in andere Zahlensysteme}\\\\
Die Dezimalzahl 338 wird ins 5er-System umgewandelt:
\begin{itemize}
    \item 338 : 5 = 67 Rest 3
    \item 67 : 5 = 13 Rest 2
    \item 13 : 5 = 2 Rest 3
    \item 2 : 5 = 0 Rest 2
    \item Rückwärts gelesen: 2323
\end{itemize}
\bigskip
\textbf{Umrechnen von anderen Zahlensystemen in Dezimalzahlen}\\\\
Die Zahl 20022 (3er-System) wird ins Dezimalsystem umgewandelt:
\begin{itemize}
    \item 2 * 3\textsuperscript{0} = 2
    \item 2 * 3\textsuperscript{1} = 6
    \item 0 * 3\textsuperscript{2} = 0
    \item 0 * 3\textsuperscript{3} = 0
    \item 2 * 3\textsuperscript{4} = 162
    \item 2 + 6 + 0 + 0 + 162 = 170
\end{itemize}
\subsection{Negative Zahlen}
\subsubsection{Einerkomplement}
\begin{itemize}
    \item 1. Die Zahl $-6$ wird ins Dualsystem umgewandelt: $6 = 0110$
    \item 2. Das Einerkomplement wird gebildet, indem alle Bits invertiert werden: $1001$
    \item 3. Das Ergebnis ist $-6$ im Einerkomplement: $1001$
    \end{itemize}
\subsubsection{Zweierkomplement}
Wertebereich z.B. 8 Bit: $+127$ bis $-128$, Asymentrie aufgrund der $0$.
\begin{itemize}
    \item 1. Subtraktion ist auch eine Addition mit einer negativen Zahl $2 - 6 = 2 + (-6) = -4$
    \item 2. Die Addition $2 + (-6)$ aufschreiben
    \item 3. Zahlen aus dem Dezimal- ins Dualsystem umschreiben. $2 = 0010 ; 6 = 0110$
    \item 4. Da wir mit einer negativen Zahl rechnen $-6$, müssen wir das Komplement ($1001$) bilden und mit $1$ ($0001$) addieren, damit wir das sogenannte Zweierkomplement erhalten.
    \item 5. Addition vom Komplement und $1$: 
    \begin{align*}
        1001\\
        +0001\\
        \hline
        1010
    \end{align*}
    \item 6. Addition mit der $2$ und $-6$ \\
    $2 + (-6)$:
    \begin{align*}
        0010\\
        1010\\
        \hline
        1100\\
        = -4
    \end{align*}
\end{itemize}
\bigskip
Kurzgesagt: Um ein Zweierkomplement zu bilden muss man invertieren und mit$ 1$ ($0001$) addieren.
