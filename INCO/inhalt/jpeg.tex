\section{JPEG}
\begin{enumerate}
	\item Y Cr Cb Conversion
	\item 8x8 Pixel Blocks
	\item Discrete Cosine Transformation (DCT)
	\item Quantization
	\item Zig-Zag Scanning
	\item DC and AC Seperation
	\item Run-Length Encoding
	\item Huffman Encoding
	\item JFIF File Creation
\end{enumerate}
\subsection{y $C_r$ $C_b$ Conversion}
Das Bild wird  in Lumineszenz (Y) und Chrominanz ($C_r$, $C_b$) aufgeteilt.
$C_b$ ist der Blauanteil und $C_r$ der Rotanteil.
\subsubsection{Subsampling}
\begin{align*}
	R = \frac{\text{Resultierende Pixel}}{\text{Ursprüngliche Pixel}} = \frac{C_r + C_b + Y}{8 * 3}
\end{align*}
\begin{itemize}
	\item Subsampling meint, dass in beiden Chrominanz-Ebenen in der
	      Horizontalen oder Vertikalen mehrere Pixel zusammengefasst werden.
	\item Der Schema-Indikator gibt die Art des Subsamplings an und hat die
	      Form J:a:b (z.B. 4:2:0)
	\item Diese Notation basiert auf einem Referenzbildblock, der J Pixel breit und
	      2 Pixel hoch ist. Üblich ist J = 4.
\end{itemize}
\includegraphics[scale=0.56]{subsampling1}\\
\includegraphics[scale=0.55]{subsampling2}

\subsection{8x8 Pixel Blocks}
Das Bild wird in 8x8 Pixel Blöcke aufgeteilt. Diese Blöcke werden dann einzeln bearbeitet.
\subsection{Discrete Cosine Transformation (DCT)}
Jeder wert der 8x8 Matrix wird durch die DCT in einen Frequenzraum transformiert.
\begin{equation*}
	F(u,v) = \frac{1}{4} \cdot C(u) \cdot C(v) \cdot \sum_{x=0}^{7} \sum_{y=0}^{7} f(x,y) \cdot \cos(\frac{(2x+1)u\pi}{16}) \cdot \cos(\frac{(2y+1)v\pi}{16})
\end{equation*}
Vor DCT:
\begin{center}
	\begin{tabular}{ c c c c c c c c c }
		139 & 144 & 149 & 153 & 155 & 155 & 155 & 155 \\
		144 & 151 & 153 & 156 & 159 & 156 & 156 & 156 \\
		150 & 155 & 160 & 163 & 158 & 156 & 156 & 156 \\
		159 & 161 & 162 & 160 & 160 & 159 & 159 & 159 \\
		159 & 160 & 161 & 162 & 162 & 155 & 155 & 155 \\
		161 & 161 & 161 & 161 & 160 & 157 & 157 & 157 \\
		162 & 162 & 161 & 163 & 162 & 157 & 157 & 157 \\
		162 & 162 & 161 & 161 & 163 & 158 & 158 & 158 \\
	\end{tabular}
\end{center}
Nach DCT ($Q_{vu}$):
\begin{itemize}
	\item \textbf{\textcolor{green}{grosser DC-Wert (Mass für Mittelwert)}} (Hoch = Weiss, Tief = Schwarz)
	\item \textbf{\textcolor{red}{wenig tieffrequente grössere AC-Werte}}
	\item \textbf{viele kleine AC-Werte (vernachlässigbar!)}
\end{itemize}
\begin{center}
	\begin{tabular}{ c c c c c c c c }
		\textcolor{green}{\textbf{1260}} & -1.0                            & \textcolor{red}{\textbf{-12.1}} & -5.2 & 2.1  & -1.7 & -2.7 & 1.3  \\
		\textcolor{red}{\textbf{-22.6}}  & \textcolor{red}{\textbf{-17.5}} & -6.2                            & -3.2 & -2.9 & -0.1 & 0.4  & -1.2 \\
		\textcolor{red}{\textbf{-10.9}}  & \textcolor{red}{\textbf{-9.3}}  & -1.6                            & 1.5  & 0.2  & -0.9 & -0.6 & -0.1 \\
		-7.1                             & -1.9                            & 0.2                             & 1.5  & 0.9  & -0.1 & -0.0 & 0.3  \\
		-0.6                             & -0.8                            & 1.5                             & 1.6  & -0.1 & -0.7 & 0.6  & 1.3  \\
		1.8                              & -0.2                            & 1.6                             & -0.3 & -0.8 & 1.5  & 1.0  & -1.0 \\
		-1.3                             & -0.4                            & -0.3                            & -1.5 & -0.5 & 1.7  & 1.1  & -0.8 \\
		-2.6                             & 1.6                             & -3.8                            & -1.8 & 1.9  & 1.2  & -0.6 & -0.4 \\
	\end{tabular}
\end{center}
\subsection{Quantization}
Die Frequenzanteile von der DCT werden nun durch eine Quantizationstabelle ($Q_{vu}$) geteilt. Die Quantizationstabelle
wird je nach JPEG Verfahren anders gewählt. Quantizationstabelle für Luminanz:
\begin{center}
	\begin{tabular}{ c c c c c c c c }
		16 & 11 & 10 & 16 & 24  & 40  & 51  & 61  \\
		12 & 12 & 14 & 19 & 26  & 58  & 60  & 55  \\
		14 & 13 & 16 & 24 & 40  & 57  & 69  & 56  \\
		14 & 17 & 22 & 29 & 51  & 87  & 80  & 62  \\
		18 & 22 & 37 & 56 & 68  & 109 & 103 & 77  \\
		24 & 35 & 55 & 64 & 81  & 104 & 113 & 92  \\
		49 & 64 & 78 & 87 & 103 & 121 & 120 & 101 \\
		72 & 92 & 95 & 98 &
	\end{tabular}
\end{center}
Nun nimmt man die 8x8 Tabelle von nach der DCT und teilt sie durch die Quantizationstabelle bzw.
Folgende Funktion:
\begin{align*}
	F_{vu} = round(F_{vu}/Q_{vu})
\end{align*}

\begin{center}
	\begin{tabular}{ c c c c c c c c }
		79 & 0  & -1 & 0 & 0 & 0 & 0 & 0 \\
		-2 & -1 & 0  & 0 & 0 & 0 & 0 & 0 \\
		-1 & -1 & 0  & 0 & 0 & 0 & 0 & 0 \\
		-1 & 0  & 0  & 0 & 0 & 0 & 0 & 0 \\
		0  & 0  & 0  & 0 & 0 & 0 & 0 & 0 \\
		0  & 0  & 0  & 0 & 0 & 0 & 0 & 0 \\
		0  & 0  & 0  & 0 & 0 & 0 & 0 & 0 \\
		0  & 0  & 0  & 0 & 0 & 0 & 0 & 0
	\end{tabular}
\end{center}
\subsection{Zig-Zag Scanning}
\begin{center}
	\includegraphics[scale=0.48]{zigzag}
\end{center}


