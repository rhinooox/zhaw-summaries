\section{Zahlensysteme}

\textbf{Umrechnen von Dezimalzahlen in andere Zahlensysteme}\\\\
Die Dezimalzahl 338 wird ins 5er-System umgewandelt:
\begin{itemize}
    \item 338 : 5 = 67 Rest 3
    \item 67 : 5 = 13 Rest 2
    \item 13 : 5 = 2 Rest 3
    \item 2 : 5 = 0 Rest 2
    \item Rückwärts gelesen: 2323
\end{itemize}
\bigskip
\textbf{Umrechnen von anderen Zahlensystemen in Dezimalzahlen}\\\\
Die Zahl 20022 (3er-System) wird ins Dezimalsystem umgewandelt:
\begin{itemize}
    \item 2 * 3\textsuperscript{0} = 2
    \item 2 * 3\textsuperscript{1} = 6
    \item 0 * 3\textsuperscript{2} = 0
    \item 0 * 3\textsuperscript{3} = 0
    \item 2 * 3\textsuperscript{4} = 162
    \item 2 + 6 + 0 + 0 + 162 = 170
\end{itemize}
\section{Zahlenmengen}

{\bf Natürliche Zahlen}\\
\(\mathbb{N} = \{0,1,2,3,4,5,6,...\}\) \\
{\bf Ganze Zahlen}\\
\(\mathbb{Z} = \{...,-3,-2,-1,0,1,2,3,...\}\) \\
{\bf Rationale Zahlen}\\
\(\mathbb{Q} = \left\{ \frac{1}{2}, \frac{2}{3}, \frac{5}{4}, -\frac{3}{7}, 0, 1, -2, ... \right\}\) \\
{\bf Reele Zahlen}\\
\(\mathbb{R} = \{ -2, 0, 1.5, \sqrt{2}, \pi, e, ... \}\)

\section{Zahlensysteme}

\section{Prädikate}
Es sei n eine natürliche Zahl. Ein Ausdruck, in dem n viele
(verschiedene) Variablen frei vorkommen und der bei Belegung (=
Ersetzen) aller freien Variablen in eine Aussage übergeht, nennen wir
ein n-stelliges Prädikat.
\begin{itemize}
    \item $x > 3$ ist ein 1-stelliges Prädikat.
    \item $x + y = z$ ist ein 3-stelliges Prädikat.
    \item x ist eine natürliche Zahl 1-stelliges Prädikat.
\end{itemize}
\subsection{Aussagen}
Aussagen sind 0-stellige Prädikate. Sie sind entweder wahr oder falsch.
\subsection{Quantoren}
\begin{itemize}
    \item \(\forall A\) (Allquantor aka für jedes Element)
    \item \(\exists A\) (Existenzquantor aka mind. ein Element)
\end{itemize}
\subsection{Junktoren}
\begin{itemize}
    \item \(A \neg B\) (Negation)
    \item \(A \wedge B\) (Konjunktion)
    \item \(A \vee B\) (Disjunktion)
    \item \(A \Rightarrow B\) (Implikation)
    \item \(A \Leftrightarrow B\) (Äquivalenz)
\end{itemize}

\section{Gesetze und Umformungen}
\begin{itemize}
    \item Distributiv:
        \begin{itemize}
            \item \(A \wedge (B \vee C) \Leftrightarrow (A \wedge B) \vee (A \wedge C)\)
            \item \(A \vee (B \wedge C) \Leftrightarrow (A \vee B) \wedge (A \vee C)\)
        \end{itemize}
    
    \item Assoziativ:
        \begin{itemize}
            \item \(A \wedge (B \wedge C) \Leftrightarrow (A \wedge B) \wedge C\)
            \item \(A \vee (B \vee C) \Leftrightarrow (A \vee B) \vee C\)
        \end{itemize}
    
    \item de Morgan:
        \begin{itemize}
            \item \(\neg (A \wedge B) \Leftrightarrow \neg A \vee \neg B\)
        \end{itemize}
\end{itemize}

\section{Aussonderung}
Ist A eine Menge und ist E(x) eine Eigenschaft (ein Prädikat), dann
bezeichnen wir mit dem Term:
\begin{equation}
    x \in A \mid E(x)
\end{equation}
Beispiel: Menge aller Geraden Zahlen:
\begin{equation}
    \{x \in \mathbb{N} \mid \exists{y} \in \mathbb{N}(x= 2y)\}
\end{equation}
\section{Ersetzung}
Ist A eine Menge und t(x) ein Ausdruck in x , dann schreiben wir
\begin{equation}
    t(A) = \{t(x) \mid x \in A\}
\end{equation}
für die Menge, die als Elemente alle Objekte von der Form t(x) mit
x \(\in\) A enthält.\\\\
Beispiel: Menge aller Quadratzahlen
\begin{equation}
    \{x^2 \mid x \in \mathbb{N}\}
\end{equation}

