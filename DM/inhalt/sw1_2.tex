\section{Zahlenmengen}
Natürliche Zahlen\\
\(\mathbb{N} = \{0,1,2,3,4,5,6,...\}\) \\
Ganze Zahlen\\
\(\mathbb{Z} = \{...,-3,-2,-1,0,1,2,3,...\}\) \\
Rationale Zahlen\\
\(\mathbb{Q} = \left\{ \frac{1}{2}, \frac{2}{3}, \frac{5}{4}, -\frac{3}{7}, 0, 1, -2, ... \right\}\) \\
Reele Zahlen\\
\(\mathbb{R} = \{ -2, 0, 1.5, \sqrt{2}, \pi, e, ... \}\)

\section{Zahlensysteme}

\section{Prädikate}
Es sei n eine natürliche Zahl. Ein Ausdruck, in dem n viele
(verschiedene) Variablen frei vorkommen und der bei Belegung (=
Ersetzen) aller freien Variablen in eine Aussage übergeht, nennen wir
ein n-stelliges Prädikat.
\begin{itemize}
    \item x > 3 ist ein 1-stelliges Prädikat.
    \item x + y = z ist ein 3-stelliges Prädikat.
    \item x ist eine natürliche Zahl 1-stelliges Prädikat.
\end{itemize}
\subsection{Aussagen}
Aussagen sind 0-stellige Prädikate. Sie sind entweder wahr oder falsch.
\subsection{Quantoren}
\(\forall A\) (Allquantor)\\
\(\exists A\) (Existenzquantor)
\subsection{Junktoren}
\(A \neg B\) (Negation)\\
\(A \wedge B\) (Konjunktion)\\
\(A \vee B\) (Disjunktion)\\
\(A \Rightarrow B\) (Implikation)\\
\(A \Leftrightarrow B\) (Äquivalenz)

\section{Gesetze und Umformungen}
Distributiv\\
\(A \wedge (B \vee C) \Leftrightarrow (A \wedge B) \vee (A \wedge C)\)\\
\(A \vee (B \wedge C) \Leftrightarrow (A \vee B) \wedge (A \vee C)\)\\

Assotiativ\\
\(A \wedge (B \wedge C) \Leftrightarrow (A \wedge B) \wedge C\)\\
\(A \vee (B \vee C) \Leftrightarrow (A \vee B) \vee C\)\\

de Morgan\\
\(\neg (A \wedge B) \Leftrightarrow \neg A \vee \neg B\)\\

\section{Aussonderung}
Ist A eine Menge und ist E (x) eine Eigenschaft (ein Prädikat), dann
bezeichnen wir mit dem Term:
\begin{align*}
    x \in A | E(x)
\end{align*}
Beispiel: Menge aller Geraden Zahlen:
\begin{align*}
    \{x \in \mathbb{N} | \exists{y} \in \mathbb{N}(x= 2y)\}
\end{align*}
\section{Ersetzung}
Ist A eine Menge und t(x) ein Ausdruck in x , dann schreiben wir
\begin{align*}
    t(A) = \{t(x) | x \in A\}
\end{align*}
für die Menge, die als Elemente alle Objekte von der Form t(x) mit
x \(\in\) A enthält.\\\\
Beispiel: Menge aller Quadratzahlen
\begin{align*}
    \{x^2 | x \in \mathbb{N}\}
\end{align*}

