\section{Mengenoperationen}
\subsection{Teilmengen}
Eine Menge A ist Teilmenge einer Menge B, geschrieben $A \subseteq B$, falls alle Elemente von A auch Elemente von B sind.
Formal gilt:
\begin{equation}
    A \subseteq B \Leftrightarrow \forall{x}(x \in A \Rightarrow x \in B)
\end{equation}
Eine Teilmenge A  von B ist eine echte Teilmenge, wenn $A \neq B$ gilt. Wir schreiben $A \subset B$,
wenn A eine echte Teilmenge von B ist.

\subsubsection{Extensionalitätsprinzip}
Mithilfe der Teilmengenrelation lässt sich das Extensionalitätsprinzip wie folgt formulieren:
\begin{equation}
    A = B \Leftrightarrow A \subseteq B \wedge B \subseteq A
\end{equation}

\subsection{Potenzmengen}
die Potenzmenge $\mathcal{P}(A)$ einer Menge A ist die Menge aller Teilmengen von A. Formal gilt:
\begin{equation}
\mathcal{P}(A) \coloneqq \{B \mid B \subseteq A\}
\end{equation}
Beispiel: 
\begin{itemize}
    \item $\mathcal{P}(\{1,2\}) = \{\emptyset, \{1\}, \{2\}, \{1,2\}\}$
    \item $\mathcal{P}(\{\{a\}\}) = \{\emptyset, \{\{a\}\}\}$
\end{itemize}

Es gilt für beliebige Mengen A:
\begin{itemize}
    \item $A \in \mathcal{P}(A)$, weil jede Megne eine Teilmenge von sich selbst ist.
    \item $\emptyset \in \mathcal{P}(A)$, weil die leere Menge Teilmenge jeder Menge ist.
\end{itemize}
{\bf! Sanity-Check}: $\mathcal{P}(A)$ hat $2^{|A|}$ Elemente.
\subsection{Verenigung}
Die Verenigung von zwei Mengen beinhaltet genau die Elemente, die in mindestens einer der beiden Mengen enthalten sind. Formal gilt:
\begin{equation}
A \cup B \coloneqq \{x \mid x \in A \vee x \in B\}
\end{equation}
Beispiel:
\begin{itemize}
    \item $\{1,2,3\} \cup \{3,4,5\} = \{1,2,3,4,5\}$
    \item $\mathbb{Z} = \{-n \mid n \in \mathbb{N} \cup \mathbb{N}\}$
\end{itemize}
\begin{itemize}
    \item Möchte man beliebig viele Mengen vereinigen, d.h. alle
    Mengen, die Element einer beliebigen Menge M von Mengen
    sind, dann ist ein Existenzquantor nötig.
    \begin{equation}
        \bigcup_{A \in M} A \coloneqq \{x \mid \exists{A} \in M(x \in A)\}
    \end{equation}
    \item Sind die Mengen die man vereinigen möchte indexiert, d.H. M ist in der Form $M = \{A_i \mid i \in I\}$,
    dann verwenden wir auch die folgenden Notation:
    \begin{equation}
        \bigcup_{i \in I} A_i \coloneqq \bigcup_{A \in M} = \{x \mid \exists{i} \in I(x \in A_i)\}
    \end{equation}
\end{itemize}

{\bf Eigenschaften von $\cup$}
\begin{itemize}
    \item Kommutativität $A \cup B = B \cup A$
    \item Assoziativität $(A \cup B) \cup C = A \cup (B \cup C)$
    \item Idempotenz $A \cup A = A$
    \item $A \subseteq A \cup B$
    \item $A \subseteq B \Leftrightarrow B = A \cup B$
\end{itemize}

\subsection{Schnittmengen}
 Die Schnittmenge von zwei Mengen beinhaltet genau die Elemente, die in beiden Mengen enthalten sind:
    \begin{equation}
        A \cap B \coloneqq \{x \mid x \in A \wedge x \in B\}
    \end{equation}
Beispiel:
\begin{itemize}
    \item $\{1,2,3\} \cap \{2,3,4,5\} = \{2,3\}$
    \item $\mathbb{N} = \{r \in \mathbb{R} \mid r \geq 0\} \cap \mathbb{Z}$
\end{itemize}
\begin{itemize}
    \item Möchte man beliebig viele Mengen schneiden, d.h. alle
    Mengen, die Element einer beliebigen Menge M von Mengen
    sind, dann ist ein Allquantor nötig.
    \begin{equation}
        \bigcap_{A \in M} A \coloneqq \{x \mid \forall{A} \in M(x \in A)\}
    \end{equation}
    \item Wenn man sie indexiert haben möchte d.H. M ist in der Form $M = \{A_i \mid i \in I\}$, dann so:
    \begin{equation}
        \bigcap_{i \in I} A_i \coloneqq \bigcap_{A \in M}A = \{x \mid \forall{i} \in I(x \in A_i)\}
    \end{equation}
\end{itemize}
{\bf Eigenschaften von $\cap$}
\begin{itemize}
    \item Kommutativität $A \cap B = B \cap A$
    \item Assoziativität $(A \cap B) \cap C = A \cap (B \cap C)$
    \item Idempotenz $A \cap A = A$
    \item $A \cap B \subseteq A$
    \item $A \subseteq B \Leftrightarrow A \cap B = A $
\end{itemize}
\subsection{Disjunkte Mengen}
\begin{itemize}
    \item zwei Mengen A und B heissen {\bf disjunkt}, wenn $A \cap B = \emptyset$ gilt.
    \item Eine Menge $M = \{A_i \mid i \in I\}$ von Mengen heissen {\bf paarweise disjunkt}, 
    wenn für alle aus $i \neq j$ gilt $A_i \cap A_j = \emptyset$ folgt.
\end{itemize}
\subsection{Differenzmengen}
Sind A und B Mengen, dann bezeichnen wir mit
\begin{equation}
    A \setminus B \coloneqq \{x \in A \mid x \notin B\}
\end{equation}
die Differenz (A ohne B ) von A und B
\subsubsection{Interaktion von $\cup$,$\cap$ und $\setminus$}
Sind A, B und C beliebige Mengen, dann gilt:
\begin{itemize}
    \item De Morgan: $C \setminus (A \cap B) = (C \setminus A) \cup (C \setminus B)$
    \item De Morgan: $C \setminus (A \cup B) = (C \setminus A) \cap (C \setminus B)$
    \item Distributivität: $A \cup (B \cap C) = (A \cup B) \cap (A \cup C)$
    \item Distributivität: $A \cap (B \cup C) = (A \cap B) \cup (A \cap C)$
\end{itemize}
\section{Relationen}
\subsection{Definition}
Eine relation von A nach B ist ein Tripel $R = (G,A,B)$ bestehend aus:
\begin{itemize}
    \item Einer (beliebigen) Menge A, genannt die Quellmenge der Relation.
    \item Einer (beliebigen) Menge B, genannt die Zielmenge der
    Relation.
    \item Einer Menge $G \subseteq A \times B$ genannt der Graph der Relation. Gilt $A = B$m dann 
    nennen wir R eine homogene Relation auf A.
\end{itemize}
\subsubsection{Notationen}
\begin{itemize}
    \item $G_r$ ist der Graph
    \item $(G,A,A)$ kann man auch als $(G,A)$ schreiben.
    \item Ist $(x,y) \in G$, dann schreiben wir auch $xRy$ und sagen x steht in Relation zu y.
\end{itemize}
\subsection{Tupel und Produktmengen}
\subsubsection{Tupel}
\begin{itemize}
    \item Ein n-Tupel ist ein Objekt von der Form ($a_1,...,a_n$)
    \item Der i-ten (für $1 \leq i \leq  n$) Eintrag $a_i$ eines Tupels $a = (a_1,...,a_n)$ bezeichnen wir auch mit $a[i]$.
\end{itemize}
Damit Tupel gleich sind müssen sie genau die gleiche innere Struktur haben.
\begin{itemize}
    \item $(1,2,3) \neq (1,3,2)$
    \item $(1,2) \neq (1,1,2)$
\end{itemize}
\subsubsection{Kartesisches Produkt}
Das kartesische Produkt von Mengen $A_1,...,A_n$, ist die Menge
aller n-Tupel mit Einträgen aus $A_1,...,A_n$

\begin{equation}
    A_1 \times ... \times A_n \coloneqq \{(a_1,...,a_n) \mid a_1 \in A_1 \wedge...\wedge a_n \in A_n\}
\end{equation}

Beispiel:
\begin{itemize}
    \item $\{1\} \times \{a,b\} = \{(1,a),(1,b)\}$
    \item $\mathbb{N}^{2} \times \{0,1\} = \{((x,y),0) \mid x \in \mathbb{N} \wedge y \in \mathbb{N}\} \cup \{((x,y),1) \mid x \in \mathbb{N} \wedge y \in \mathbb{N}\} $
\end{itemize}
\subsubsection{Projektionen}
Ist A eine Menge von n-Tupeln und ist $k \leq n$ eine natürliche Zahl,
dann nennen wir die Menge
\begin{equation}
    pr_{k}(A) = \{x[k] \mid x \in A\}
\end{equation}
die k-te Projektion von A.\\\\
Beispiel:
\begin{itemize}
    \item $pr_{1}(\{(a,b)\}) = \{a\}$
    \item $pr_{1}(\{(1,2),(2,7),(1,5)\}) = \{1,2\}$
    \item $pr_{2}(\{(1,2),(2,7),(1,5)\}) = \{2,7,5\}$
\end{itemize}
\subsection{Darstellung von Relationen}
\subsubsection{Gerichteter Graph}
\NewAdigraph{gerichtet}{
1:0,0;
2:2,2;
3:2,-1;
4:3,0}{1,1; 1,2;
1,3; 1,4; 2,4; 2,2; 4,4; 3,3;}
\gerichtet{}
\\$xRy: \Leftrightarrow x$ teilt $y$
\subsubsection{Domain}
Es sei $R = (G,A,B)$ eine Relation.
\begin{itemize}
    \item Die Domäne von R entpsricht der Projektion auf die erste Komponente vom Graph von R:
    \begin{equation}
        \text{dom}(R) = pr_{1}(G_R)
    \end{equation}
    \item Ist die Relation R als gerichteter Graph dargestellt, dann
    entspricht die Domäne der Menge aller Punkte, von denen
    mindestens ein Pfeil ausgeht.
\end{itemize}
\subsubsection{Image}
Es sei $R = (G,A,B)$ eine Relation. Die Bildmenge einer Relation R besteht aus den Elementen aus der Ziemlenge
welche zu  mind. einem Element aus der Quellmenge in Relation stehen:
\begin{equation}
    \text{im}(R) \eqqcolon \{b \in B \mid \exists{a} \in A (aRb)\}
\end{equation}

\subsection{Klassifizierung von Relationen} 
\subsubsection{Reflexivität}
Eine (homogene) Relation R auf A heisst reflexiv, wenn jedes
Element (aus der Quellmenge) mit sich selbst in Relation steht:
\begin{equation}
    \forall{x} \in A(xRx)
\end{equation}
\NewAdigraph{reflexiv}{
1:0,0;
2:1.5,1;
3:1.5,-1
}
{
1,1;
2,2;
3,3;
1,2;
2,3
}
\reflexiv{}
\subsubsection{Symmetrie}
Eine (homogene) Relation R auf A ist symetrisch, falls:
\begin{equation}
    \forall{x,y} (xRy \Rightarrow yRx)
\end{equation}
\NewAdigraph{symetrisch}{
1:0,0;
2:1.5,-0.5;
3:3,0
}
{
1,2;
2,1;
2,3;
3,2;
1,1;
2,2;
3,3
}
\symetrisch{}
\subsubsection{Antisymmetrie}
Eine (homogene) Relation R auf A ist antisymetrisch, falls:
\begin{equation}
    \forall{x,y} (xRy \wedge yRx \Rightarrow x = y)
\end{equation}
\NewAdigraph{antisymmetrisch}{
1:0,0;
2:1.5,-0.5;
3:3,0
}
{
2,1;
1,1;
2,3;
}
\antisymmetrisch{}
\\Ein Graph kann symmetrisch, antisymmetrisch, weder noch, oder beides zusammen sein.
\subsubsection{Transitivität} 
Eine (homogene) Relation R auf einer Menge A heisst transitiv,
falls
\begin{equation}
    \forall{x,y,z} (xRy \wedge yRz \Rightarrow xRz)
\end{equation}
gilt.\\\\
Ein Graph ist transitiv, wenn jede ''Abkürzung'' drin ist:\\
\NewAdigraph{transitiv}{
1:0,0;
2:1.5,1;
3:1.5,-1;
4:3,0
}
{
1,1;
2,2;
3,3;
4,4;
1,2;
2,4;
1,4;
}
\transitiv{}
\subsubsection{linksvollständig / linkstotal}
Für  $R = (G,A,B)$...
    \begin{equation}
        A = dom(R)
    \end{equation}
\NewAdigraph{linkstotal}{
    A1:0,1;
    A2:0,0;
    A3:0,-1;
    B1:2,0.5;
    B2:2,-0.5
    }
    {
    A2,B1;
    A1,B1;
    A3,B1
    }
\linkstotal{}
\subsubsection{rechtsvollständig / rechtstotal}
Für  $R = (G,A,B)$...
    \begin{equation}
        B = im(R)
    \end{equation}
    \NewAdigraph{rechtstotal}{
        A1:0,1;
        A2:0,0;
        A3:0,-1;
        B1:2,0.5;
        B2:2,-0.5
        }
        {
        A3,B2;
        A1,B1;
        A3,B2
        }
    \rechtstotal{}
\subsubsection{linkseindeutig}
Für  $R = (G,A,B)$...
    \begin{equation}
        \forall{x_1,x_2,y}(x_1Ry \wedge x_2Ry \Rightarrow x_1 = x_2)
    \end{equation}
\NewAdigraph{linkseindeutig}{
    A1:0,1;
    A2:0,0;
    A3:0,-1;
    B1:2,1.5;
    B2:2,0.5;
    B3:2,-0.5;
    B4:2,-1.5
    }
    {
    A1,B2;
    A2,B1;
    A2,B3;
    A3,B4
    }
\linkseindeutig{}
\subsubsection{rechtseindeutig}
Für  $R = (G,A,B)$...
    \begin{equation}
        \forall{x,y_1,y_2}(xRy_1 \wedge xRy_2 \Rightarrow y_1 = y_2)
    \end{equation}
    \NewAdigraph{rechtseindeutig}{
        A1:0,1;
        A2:0,0;
        A3:0,-1;
        B1:2,1.5;
        B2:2,0.5;
        B3:2,-0.5;
        B4:2,-1.5
        }
        {
        A1,B2;
        A2,B2
        }
    \rechtseindeutig{}
