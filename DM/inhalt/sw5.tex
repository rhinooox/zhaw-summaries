\section{Funktionen $\rightarrow$}
Damit eine Relation eine Funktionen ist, muss sie folgende Eigenschaften haben:
\begin{itemize}
    \item \textbf{rechtseindeutig}
    \item \textbf{linksvollständig}
\end{itemize}
\subsection{Injektive Funktionen $\hookrightarrow$}
Damit eine Funktionen injektiv ist muss sie folgende Eigenschaften haben:
\begin{itemize}
    \item \textbf{linksvollständig}
    \item \textbf{rechtseindeutig}
    \item \textbf{linkseindeutig}
\end{itemize}
Eine Funktionen $f : A \rightarrow B$ heisst injektiv, falls unterschiedliche Ipunts stets in
unterschiedlichen Outputs resultieren:
\begin{align*}
    \forall{x_1, x_2} \in A(x_1 \neq x_2 \Rightarrow f(x_1) \neq f(x_2))
\end{align*}
\begin{center}
    \NewAdigraph{injektiv}{
        A1:0,1;
        A2:0,0;
        A3:0,-1;
        B1:2,0.5;
        B2:2,-0.5
        }
        {
        A1,B1;
        A2,B1;
        A3,B2
        }
    \injektiv{}
\end{center}
\subsection{Surjektive Funktionen $\twoheadrightarrow$}
Damit eine Funktionen injektiv ist muss sie folgende Eigenschaften haben:
\begin{itemize}
    \item \textbf{linksvollständig}
    \item \textbf{rechtseindeutig}
    \item \textbf{rechtsvollständig}
\end{itemize}
Eine Funktion $f = (G,A,B)$ heisst surjektiv, falls $im(f) = B$ gilt.
\begin{center}
    \NewAdigraph{surjektiv}{
        A1:0,1;
        A2:0,0;
        B1:2,1;
        B2:2,0;
        B3:2,-1
        }
        {
        A1,B1;
        A2,B2;
        }
    \surjektiv{}
\end{center}
\subsection{Bijektive Funktionen $\rightleftharpoons$}
Damit eine Funktionen injektiv ist muss sie folgende Eigenschaften haben:
\begin{itemize}
    \item \textbf{linksvollständig}
    \item \textbf{rechtsvollständig}
    \item \textbf{rechtseindeutig}
    \item \textbf{linkseindeutig}  
\end{itemize}
Oder anders gesagt: Eine Funktion $f : A \rightarrow B$ heisst bijektiv, 
falls sie sowohl injektiv als auch surjektiv ist.
\subsection{Umkehrfunktionen}
Für die Umkehrfunktionen einfach nach x auflösen und dann x und y vertauschen.\\
Eigenschaften von Umkehrfunktionen:
\begin{itemize}
    \item Für jede Relation R gilt $R^{-1^{-1}} = R$
    \item R ist genau dann linksvollständig, wenn $R^{-1}$ rechtseindeutig ist.
    \item R ist genau dann linkseindeutig, wenn $R^{-1}$ rechtseindeutig ist.
\end{itemize}
\subsection{Komposition}
Für $g: A \rightarrow B $ und $f: B \rightarrow C$ definieren wir:
\begin{align*}
    f \circ g: A \rightarrow C
\end{align*}
\begin{align*}
    (f \circ g)(x) = f(g(x))
\end{align*}
Wörtlich sagt man auch ''f nach g'' da f nach g ausgeführt wird bzw. g zuerst ausgeführt wird.
\subsubsection{Assoziativität}
Für $f: A \rightarrow B$, $g: B \rightarrow C$ und $h: C \rightarrow D$ gilt:
\begin{itemize}
    \item $(f \circ g) \circ h = f \circ (g \circ h)$
\end{itemize}

\subsubsection{Injektivität, Surjektivität und Komposition}
Es seien $f: A \rightarrow B$ und $g: B \rightarrow C$ Funktionen.
\begin{itemize}
    \item Sind f und g injektiv, so ist auch $g \circ f : A \rightarrow C$ injektiv.
    \item Sind f und g surjektiv, so ist auch $g \circ f : A \rightarrow C$ surjektiv.
    \item Sind f und g bijektiv, so ist auch $g \circ f : A \rightarrow C$ bijektiv.
\end{itemize}