\section{Funktionen $\rightarrow$}
Damit eine Relation eine Funktionen ist, muss sie folgende Eigenschaften haben:
\begin{itemize}
    \item \textbf{rechtseindeutig}
    \item \textbf{linksvollständig}
\end{itemize}
\subsection{Injektive Funktionen $\hookrightarrow$}
Damit eine Funktionen injektiv ist muss sie folgende Eigenschaften haben:
\begin{itemize}
    \item \textbf{linksvollständig}
    \item \textbf{rechtseindeutig}
    \item \textbf{linkseindeutig}
\end{itemize}
Eine Funktionen $f : A \rightarrow B$ heisst injektiv, falls unterschiedliche Ipunts stets in
unterschiedlichen Outputs resultieren:
\begin{equation}
    \forall{x_1, x_2} \in A(x_1 \neq x_2 \Rightarrow f(x_1) \neq f(x_2))
\end{equation}
\newline
\NewAdigraph{injektiv}{
    A1:0,1;
    A2:0,0;
    A3:0,-1;
    B1:2,0.5;
    B2:2,-0.5
    }
    {
    A1,B1;
    A2,B1;
    A3,B2
    }
\injektiv{}
\newline
\newline
\subsection{Surjektive Funktionen $\twoheadrightarrow$}
Damit eine Funktionen injektiv ist muss sie folgende Eigenschaften haben:
\begin{itemize}
    \item \textbf{linksvollständig}
    \item \textbf{rechtseindeutig}
    \item \textbf{rechtsvollständig}
\end{itemize}
Eine Funktion $f = (G,A,B)$ heisst surjektiv, falls $im(f) = B$ gilt.
\newline
\newline
\NewAdigraph{surjektiv}{
    A1:0,1;
    A2:0,0;
    B1:2,1;
    B2:2,0;
    B3:2,-1
    }
    {
    A1,B1;
    A2,B2;
    }
\surjektiv{}
\newline
\subsection{Bijektive Funktionen $\rightleftharpoons$}
Damit eine Funktionen injektiv ist muss sie folgende Eigenschaften haben:
\begin{itemize}
    \item \textbf{linksvollständig}
    \item \textbf{rechtsvollständig}
    \item \textbf{rechtseindeutig}
    \item \textbf{linkseindeutig}  
\end{itemize}
Oder anders gesagt: Eine Funktion $f : A \rightarrow B$ heisst bijektiv, 
falls sie sowohl injektiv als auch surjektiv ist.
\subsection{Umkehrfunktionen}
Für die Umkehrfunktionen einfach nach x auflösen und dann x und y vertauschen.\\
Eigenschaften von Umkehrfunktionen:
\begin{itemize}
    \item Für jede Relation R gilt $R^{-1^{-1}} = R$
    \item R ist genau dann linksvollständig, wenn $R^{-1}$ rechtseindeutig ist.
    \item R ist genau dann linkseindeutig, wenn $R^{-1}$ rechtseindeutig ist.
\end{itemize}
\subsection{Komposition}
Für $g: A \rightarrow B $ und $f: B \rightarrow C$ definieren wir:
\begin{equation}
    f \circ g: A \rightarrow C
\end{equation}
\begin{equation}
    (f \circ g)(x) = f(g(x))
\end{equation}
Wörtlich sagt man auch ''f nach g'' da f nach g ausgeführt wird bzw. g zuerst ausgeführt wird.
\subsubsection{Assoziativität}
Für $f: A \rightarrow B$, $g: B \rightarrow C$ und $h: C \rightarrow D$ gilt:
\begin{itemize}
    \item $(f \circ g) \circ h = f \circ (g \circ h)$
\end{itemize}