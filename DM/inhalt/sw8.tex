\section{Unendliche Mengen}
Eine Menge $A$ ist nicht endlich, wenn $|A| = \infty$
\subsection{Abzählbare Mengen}
Eine Menga $A$ heisst abzählbar, wenn $A = \emptyset$ oder es eine der folgenden (äquivalenten)
Bedingungen erfüllt:
\begin{itemize}
	\item $|A| = |\N|$
	\item Es gibt eine surjektive Funktion $f: \N \twoheadrightarrow A$
	\item Es gibt eine injektive Funktion $g: A \hookrightarrow \N$
\end{itemize}
Beispiele sind:
\begin{itemize}
	\item $\{1,2,3\}$
	\item $\N$
	\item $\Z$
	\item $\Q$
	\item $\PR$
\end{itemize}
\subsection{Überabzählbare Mengen}
Überabzählbare Mengen sind nicht abzählbar.
Beispiele sind:
\begin{itemize}
	\item $\R$: reele Zahlen
	\item $\C$: komplexe Zahlen
	\item $\I$: imaginäre Zahlen
	\item $B(\infty)$: alle unendlichen Binärsequenzen
	\item $\poset(\N)$: Potenzmenge von $\N$
\end{itemize}
\subsection{Satz von Canton-Bernstein}
Für beliebige nichtleere Mengen $A$ und $B$ sind folgende Aussagen
äquivalent:
\begin{itemize}
	\item $|A| \leq |B| \wedge |B| \leq |A|$
	\item $|A| = |B|$
\end{itemize}
\subsubsection{Schubfachprinzip}
Aus $|A| > |B|$ und $|A| \neq |B|$ folgt $|B| \not\leq |A|$
\subsubsection{Definition von Dedekind}
Eine Menge $A$ ist genau dann unendlich, wenn es eine injektive und nicht surjektive Funktion
\\$f: A \hookrightarrow A$ gibt.
	\subsubsection{Hilberts Hotel}
	Eine Menga $A$ ist genau dann unendlich, wenn eine echte Teilmenge $B \subset A$ mit $|A| = |B|$ existiert.