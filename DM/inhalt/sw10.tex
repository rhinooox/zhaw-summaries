\section{Formale Aussagenlogik}

\subsection{Formale Definition}
Die Syntax der Aussagenlogik (= Menge aller aussagenlogischen Formeln) ist
durch $N(\{x_1,x_2,...\},\{and,or,not\})$ mit \dots
\begin{align*}
	and(A,B) := A \vee B  \\
	or(A,B) := A \wedge B \\
	not(A) := \neg A
\end{align*}
\dots gegeben.
\subsubsection{Beispiele}
\begin{itemize}
	\item $x_2$
	\item $((x_1 \wedge x_7) \vee (x_5 \wedge \neg x_0))$
	\item $\neg\neg\neg(x_3 \wedge \neg x_5)$
\end{itemize}
\subsection{Visualisierung}

\begin{align*}
	(x_1 \wedge x_7) \vee (x_5 \wedge \neg x_0)
\end{align*}

\begin{center}
	\begin{tikzpicture}
		[level distance=1.5cm,
			level 1/.style={sibling distance=4cm},
			level 2/.style={sibling distance=2cm}]
		\node {$\vee$}
		child {node {$\wedge$}
				child {node {$x_1$}}
				child {node {$x_7$}}
			}
		child {node {$\wedge$}
				child {node {$x_5$}}
				child {node {$\neg$}
						child {node {$x_0$}}
					}
			};
	\end{tikzpicture}
\end{center}
\subsection{Strukturelle Rekursion}
\begin{itemize}
    \item $\wedge = min()$
    \item $\vee = max()$
\end{itemize}
\begin{align*}
    \llbracket(x_1 \wedge x_2) \vee x_3\rrbracket_3(1, 1, 0) &= \max(\llbracket x_1 \wedge x_2\rrbracket_3(1, 1, 0), \llbracket x_3\rrbracket_3(1, 1, 0)) \\
    &= \max(\min(\llbracket x_1\rrbracket_3(1, 1, 0), \llbracket x_2\rrbracket_3(1, 1, 0)), 0) \\
    &= \max(\min(1, 1), 0) \\
    &= \max(1, 0) \\
    &= 1
\end{align*}
\subsection{Normalformen}
\subsubsection{Definition}
\begin{itemize}
\item $K_0 := D_0 := \{x_1, x_2 \ldots \} \cup \{\neg x_1, \neg x_2 \ldots \}$
\item $K_{n+1} := \{A_1 \wedge \cdots \wedge A_k \mid A_1 \ldots A_k \in D_n\}$
\item $D_{n+1} := \{A_1 \vee \cdots \vee A_k \mid A_1 \ldots A_k \in K_n\}$
\end{itemize}

\subsubsection{Beispiele}
\begin{itemize}
\item $(\neg x_1 \wedge x_2) \vee x_3 \in D_2$
\item $(\neg x_1 \vee x_2) \wedge (x_3 \vee x_1) \in K_2$
\end{itemize}

\subsubsection{Bemerkung}
Es gelten für alle $n \in \mathbb{N}$:
\begin{itemize}
\item $D_n \subsetneq K_{n+1} \cap D_{n+1}$
\item $K_n \subsetneq D_{n+1} \cap K_{n+1}$
\end{itemize}
\subsection{KNF und DNF}
\subsubsection{Definition}
Die Formeln in $K_2$ sind in \textbf{konjunktiver Normalform (KNF)}, die
Formeln in $D_2$ sind in \textbf{disjunktiver Normalform (DNF)}.
\subsubsection{Satz}
Zu jeder Formel $A$ existieren äquivalente Formeln $A_K$ in KNF und
$A_D$ in DNF.
