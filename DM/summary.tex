\documentclass[a4paper,10pt]{article}
\usepackage[utf8]{inputenc}
\usepackage{amsmath, amssymb}
\usepackage{graphicx}
\usepackage{hyperref}
\usepackage{geometry}
\geometry{margin=1in}

\title{Diskrete Mathematik}
\author{Cédric Volk}
\date{\today}

\begin{document}

\maketitle

\tableofcontents
\newpage
\section{Zahlenmengen}
\subsection{Natürliche Zahlen}
\(\mathbb{N} = \{0,1,2,3,4,5,6,...\}\)

\subsection{Ganze Zahlen}
\(\mathbb{Z} = \{...,-3,-2,-1,0,1,2,3,...\}\)
\subsection{Rationale Zahlen}
\(\mathbb{Q} = \left\{ \frac{1}{2}, \frac{2}{3}, \frac{5}{4}, -\frac{3}{7}, 0, 1, -2, ... \right\}\)
\subsection{Reelle Zahlen}
\(\mathbb{R} = \{ -2, 0, 1.5, \sqrt{2}, \pi, e, ... \}\)

\section{Zahlensysteme}

\section{Prädikate}
\subsection{Aussagen}
\subsection{Quantoren}
\subsubsection{Allquantor}
\(\forall A\)
\subsubsection{Existenzquantor}
\(\exists A\)
\subsection{Junktoren}
\subsubsection{Negation}
\(A \neg B \)
\subsubsection{Konjunktion}
\(A \wedge B\)
\subsubsection{Disjunktion}
\( \vee A\)
\subsubsection{Implikation}
\(B \Rightarrow A\)
\subsubsection{Äquivalenz}
\(B \Leftrightarrow A\)


\section{Lemmas}
\subsection{Lemma 1 - Transitivität der Implikation}
Für alle Prädikate mit A,B und C mit \(A \Rightarrow B\) und \(B \Rightarrow C\) gilt \(A \Rightarrow C\).
\subsection{Lemma 2 - Kontraposition}
Für alle Prädikate mit A und B gilt \(A \Rightarrow B \Leftrightarrow \neg B \Rightarrow \neg A\).\\
Beweis. Wir wenden die Junktorenregeln an:\\
\begin{align*}
    A \Rightarrow B \\
    \Leftrightarrow \neg A \vee B && \text{Definition von A $\rightarrow$ B}\\
    \Leftrightarrow B \vee \neg A && \text{Kommutativität}\\
    \Leftrightarrow \neg\neg B \vee \neg A && \text{Doppelte Negation}\\
    \Leftrightarrow \neg B \Rightarrow \neg A && \text{Definition von $\neg B \rightarrow \neg A$}\\
\end{align*}


\end{document}