\documentclass[a4paper,10pt]{article}
\usepackage[utf8]{inputenc}
\usepackage{amsmath, amssymb}
\usepackage{graphicx}
\usepackage{hyperref}
\usepackage{geometry}
\geometry{margin=1in}

\title{Diskrete Mathematik}
\author{Cédric Volk}
\date{\today}

\begin{document}

\maketitle

\tableofcontents

\section{Introduction}
This is the introduction section.

\section{Zahlenmengen}
\subsection{Natürliche Zahlen}
\(\mathbb{N} = \{0,1,2,3,4,5,6,...\}\)
\(
\mathbb{N} = \left\{ \sum_{i=0}^{n} i \mid n \in \mathbb{N} \right\}
\)
\subsection{Ganze Zahlen}
\(\mathbb{Z} = \{...,-3,-2,-1,0,1,2,3,...\}\)
\subsection{Rationale Zahlen}
\(\mathbb{Q} = \left\{ \frac{1}{2}, \frac{2}{3}, \frac{5}{4}, -\frac{3}{7}, 0, 1, -2, ... \right\}\)
\subsection{Reelle Zahlen}
\(\mathbb{R} = \{ -2, 0, 1.5, \sqrt{2}, \pi, e, ... \}\)

\section{Formulas}

\section{Examples}
\subsection{Example 1}
Description and solution of example 1.

\subsection{Example 2}
Description and solution of example 2.

\end{document}