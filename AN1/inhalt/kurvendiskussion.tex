\section{Kurvendiskussion}
\subsection{Wendepunkte und Sattelpunkte}
\begin{align*}
	f''(x_0) = 0 \text{ und } f'''(x_0) \neq 0 \Rightarrow \textbf{Wendepunkt}
\end{align*}
\begin{align*}
	f'(x_0) = 0 \text{ und } f''(x_0) = 0 \text{ und } f'''(x_0) = 0 \Rightarrow \textbf{Sattelpunkt}
\end{align*}
\subsection{Monotonie}
Mithilfe dieses Satzes lassen sich monotone Abschnitte von Funktionen bestimmen:
\begin{align*}
	f'(x) & \text{ ist auf einem Intervall überall } \geq 0 \Leftrightarrow f \text{ ist auf diesem Intervall monoton steigend.} \\
	f'(x) & \text{ ist auf einem Intervall überall } \leq 0 \Leftrightarrow f \text{ ist auf diesem Intervall monoton fallend.}
\end{align*}
\subsection{Fragenkatalog}
\begin{enumerate}
	\item Defnitionsbereich?
	\item Symmetrieeigenschaften (gerade/ungerade), Periode?
	\item Schnittpunkte mit Achsen, Polstellen?
	\item Randpunkte bzw. Verhalten, wenn $x$ gegen die Grenzen des Defnitionsbereichs strebt?
	\item Kandidaten für Extrema bestimmen und untersuchen
	\item Wendepunkte suchen
	\item Tabelle von Werten aufstellen (falls noch nötig)
\end{enumerate}
\subsection{Extremwertaufgaben}
\begin{enumerate}
	\item Zielgrösse identifizieren.
	\item Unabhängige Variable identifizieren.
	\item Definitionsbereich bestimmen.
	\item Zielgrösse als Funktion der unabhängigen Variablen ausdrücken; ev. eine qualitative Skizze des Graphen machen.
	\item Relative Maxima resp. Minima bestimmen; Randpunkte auch berücksichtigen!
	\item Untersuchen, welche der relativen Extrema auch absolute Extrema sind (inklusive bei offenen und halboffenen Intervallen – Betrachtung der Funktion in der Nähe des Randes)
	\item Die gesuchte Information aus den Berechnungen extrahieren. (Ev. nachschauen, nach welcher Grösse gefragt wurde: Extremalstelle? Extremalwert? Extremalpunkt?)
\end{enumerate}