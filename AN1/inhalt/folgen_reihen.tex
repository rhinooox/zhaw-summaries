\section{Folgen und Reihen}
\subsection{Folgen}
\subsubsection{Arithmetische Folge}
\begin{align*}
	a_n & = c + (n-1) \cdot d
\end{align*}
\subsubsection{Geometrische Folge}
\begin{align*}
	a_n & = c \cdot q^{n-1}
\end{align*}
\subsubsection{Grenzwert von Folgen}
\textbf{Fall 1}: Zählergrad < Nennergrad. Dann gilt:
\begin{align*}
	\lim_{n \to \infty} \frac{g(n)}{h(n)} & = 0
\end{align*}
\begin{align*}
	\textcolor{red}{\lim_{n \to \infty} \frac{3n^2 + 7n - 15}{n^3 - 2n^2 + n + 10} = 0}
\end{align*}
\textbf{Fall 2}: Zählergrad > Nennergrad. Dann gilt:
\begin{align*}
	\lim_{n \to \infty} \frac{g(n)}{h(n)} & = \infty \text{ oder } -\infty
\end{align*}
\begin{align*}
	\textcolor{red}{\lim_{n \to \infty} \frac{3n^4 + 7n - 15}{6n^3 - 2n^2 + 10} \to \infty}
\end{align*}
\textbf{Fall 3}: Zählergrad = Nennergrad. Dann gilt:
\begin{align*}
	\lim_{n \to \infty} \frac{g(n)}{h(n)} & = \frac{\text{führender Term von } g}{\text{führender Term von } h}
\end{align*}
\begin{align*}
	\textcolor{red}{\lim_{n \to \infty} \frac{2n^3 + n^2 + 8n}{5n^3 + 4n^2 + 17} = \frac{2}{5}}
\end{align*}
\textbf{Spezialfall}: Folge führt gegen $e \approx 2.718$:
\begin{align*}
	((1+\frac{1}{n})^n) & = e
\end{align*}

\subsection{Rechnen mit Grenzwerten}
\begin{align*}
	\lim_{n \to \infty} c \cdot a_n & = c \cdot \lim_{n \to \infty} a_n
\end{align*}
\begin{align*}
	\lim_{n \to \infty} (a_n + b_n) & = \lim_{n \to \infty} a_n + \lim_{n \to \infty} b_n
\end{align*}
\begin{align*}
	\lim_{n \to \infty} (a_n \cdot b_n) & = \lim_{n \to \infty} a_n \cdot \lim_{n \to \infty} b_n
\end{align*}
\begin{align*}
	\lim_{n \to \infty}(\frac{a_n}{b_n}) & = \frac{\lim_{n \to \infty} a_n}{\lim_{n \to \infty} b_n}
\end{align*}
\subsubsection{Arithmetische Reihe}
\begin{align*}
    s_n & =n \cdot a_1 + \frac{n \cdot (n-1)}{2} \cdot d
\end{align*}
\subsubsection{Geometrische Reihe}
\begin{align*}
    s_n & = \frac{a_1 \cdot (1-q^n)}{1-q}
\end{align*}
\subsubsection{Grenzwert von Reihen}
\textbf{Arithmetische Reihe}\\
Geht (divergiert) immer gegen $\infty$ oder $-\infty$
\\
\\
\textbf{Geometrische Reihe}\\
Eine geometrische Reihe konvergiert genau dann, wenn $|q| < 1$ ist.
\begin{flushleft}
    \textit{\textbf{Fall 1}}: $q > 1$ \\
    Die Reihe strebt gegen $\infty$ oder $-\infty$. \\[0.5em]
    \textit{\textbf{Fall 2}}: $q \leq -1$ \\
    Die Reihe springt zwischen positiven und negativen Werten hin und her. \\[0.5em]
    \textit{\textbf{Fall 3}}: $|q| < 1$ \\
    Die Reihe strebt gegen $\frac{a_1}{1-q}$.
\end{flushleft}
