\section{Integral}
Um die Fläche unter einer Funktion zu berechnen muss man folgende Schritte durchgehen:
\begin{enumerate}
    \item Bestimme die Stammfunktion $F(x)$
    \item Berechne $F(b) - F(a)$
\end{enumerate}
\subsection{Stammfunktion}
Die Stammfunktion $F(x)$ einer Funktion $f(x)$ ist die Funktion, deren Ableitung $f(x)$ ist, also "aufleiten".
\subsection{Satz}
Gegeben ist eine Funktion $f$, die auf einem Intervall $I$ stetig ist, und eine beliebige Stamm-
funktion $F$ von $f$. Dann gilt für alle $a, b \in I$:
\begin{align*}
    \int_a^b f(x)\,dx = F(b) - F(a)
\end{align*}
\subsection{Integrale von bestimmten Funktionen}
\subsubsection{Potenz- und Logharithmusfunktionen}
\begin{itemize}
    \item $\int a^x dx = \frac{a^x}{ln(a) + C}$
    \item $\int ln(x) dx = x \cdot ln(x) - x + C$
    \item $\int log_a(x) dx = \frac{1}{ln(a)} \cdot (x \cdot ln(x) -x) + C$
\end{itemize}
\subsubsection{Trigonometrische Funktionen}
\begin{itemize}
    \item $\int \sin(x) \, dx = -\cos(x) + C$
    \item $\int \cos(x) \, dx = \sin(x) + C$
    \item $\int \tan(x) \, dx = -\ln|\cos(x)| + C$
    \item $\int (1 + \tan^2(x)) \, dx = \int \frac{1}{\cos^2(x)} \, dx = \tan(x) + C$
    \item $\int (1 - x^2)^{-1/2} \, dx = \arcsin(x) + C$
    \item $\int -(1 - x^2)^{-1/2} \, dx = \arccos(x) + C$
    \item $\int (1 + x^2)^{-1} \, dx = \arctan(x) + C$
\end{itemize}